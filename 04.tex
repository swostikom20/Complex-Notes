\chapter{Cauchy Goursat Theorem}
The Cauchy-Goursat Theorem, which as the name of the chapter suggests, is central to this chapter. So we begin by first stating the theorem itself. 
\begin{restatable}[\textbf{Cauchy-Goursat}]{thm}{}\label{}
If a function $f$ is analytic at all points interior to and on a simple-closed contour $C$ then
	$\int_{{C}}^{{}} {f\left( z \right) } \: d{z} {} = 0$. 
\end{restatable}	
This theorem is rather hard to prove, so we will be building tools to deal with it for the remainder of the chapter.\\
Before moving ahead, we would like to define what a \emph{simply connected} region is.
\begin{defn}[\textbf{Simply-connected Region}]
A domain $D$ is said to be \emph{simply connected} if every simple closed contour within it is encloses points of $D$ only. \\
A domain $D$ is said to be \emph{multiply connected} if it is not simply connected. 
\end{defn}
A simple example for simply connected region would be $\mathds{C}$ itself. Any curve $\{\gamma\} \in \mathds{C}$ can only contain points of $\mathds{C}$ itself. \\
Consider the space $\mathds{C}^* = \mathds{C} \setminus \{0\} $, it is multiply connected.

\begin{center}
		 

\tikzset{every picture/.style={line width=0.75pt}} %set default line width to 0.75pt        

\begin{tikzpicture}[x=0.75pt,y=0.75pt,yscale=-1,xscale=1]
%uncomment if require: \path (0,300); %set diagram left start at 0, and has height of 300

%Shape: Axis 2D [id:dp7089388061410618] 
\draw  (218,223.2) -- (514,223.2)(247.6,27) -- (247.6,245) (507,218.2) -- (514,223.2) -- (507,228.2) (242.6,34) -- (247.6,27) -- (252.6,34)  ;
%Shape: Polygon Curved [id:ds6075001286607652] 
\draw   (204,186) .. controls (224,176) and (314,166) .. (294,186) .. controls (274,206) and (274,216) .. (294,246) .. controls (314,276) and (224,276) .. (204,246) .. controls (184,216) and (184,196) .. (204,186) -- cycle ;
%Shape: Circle [id:dp9295466509908139] 
\draw   (242.49,223.2) .. controls (242.49,220.38) and (244.78,218.09) .. (247.6,218.09) .. controls (250.42,218.09) and (252.71,220.38) .. (252.71,223.2) .. controls (252.71,226.02) and (250.42,228.31) .. (247.6,228.31) .. controls (244.78,228.31) and (242.49,226.02) .. (242.49,223.2) -- cycle ;

% Text Node
\draw (309,158.4) node [anchor=north west][inner sep=0.75pt]    {$\gamma $};
% Text Node
\draw (406,45.4) node [anchor=north west][inner sep=0.75pt]    {$\mathbb{C} \setminus \{0\}$};


\end{tikzpicture}
	\end{center}

It is apparent that the curve contains $0 \not\in \mathds{C} \setminus \{0\}  $\\

Let $\gamma_z:= \text{OAB}$. \\
Define 
$ F(z)  = \int_{\gamma_z}^{} f(z) \: d{z}$\\
Then, we have, 



\begin{center}
\tikzset{every picture/.style={line width=0.75pt}} %set default line width to 0.75pt        

\begin{tikzpicture}[x=0.75pt,y=0.75pt,yscale=-1,xscale=1]
%uncomment if require: \path (0,300); %set diagram left start at 0, and has height of 300

%Shape: Circle [id:dp4927592098027098] 
\draw   (205,149) .. controls (205,75.55) and (264.55,16) .. (338,16) .. controls (411.45,16) and (471,75.55) .. (471,149) .. controls (471,222.45) and (411.45,282) .. (338,282) .. controls (264.55,282) and (205,222.45) .. (205,149) -- cycle ;
%Straight Lines [id:da4582551222426777] 
\draw    (338,149) -- (442,149) ;
\draw [shift={(444,149)}, rotate = 180] [color={rgb, 255:red, 0; green, 0; blue, 0 }  ][line width=0.75]    (10.93,-3.29) .. controls (6.95,-1.4) and (3.31,-0.3) .. (0,0) .. controls (3.31,0.3) and (6.95,1.4) .. (10.93,3.29)   ;
%Straight Lines [id:da9233857830894384] 
\draw    (444,149) -- (443.03,81) ;
\draw [shift={(443,79)}, rotate = 89.18] [color={rgb, 255:red, 0; green, 0; blue, 0 }  ][line width=0.75]    (10.93,-3.29) .. controls (6.95,-1.4) and (3.31,-0.3) .. (0,0) .. controls (3.31,0.3) and (6.95,1.4) .. (10.93,3.29)   ;
%Straight Lines [id:da14858467044751889] 
\draw    (391,149) -- (391,112) ;
\draw [shift={(391,110)}, rotate = 90] [color={rgb, 255:red, 0; green, 0; blue, 0 }  ][line width=0.75]    (10.93,-3.29) .. controls (6.95,-1.4) and (3.31,-0.3) .. (0,0) .. controls (3.31,0.3) and (6.95,1.4) .. (10.93,3.29)   ;

% Text Node
\draw (340,152) node [anchor=north west][inner sep=0.75pt]   [align=left] {$\displaystyle 0$};
% Text Node
\draw (446,152) node [anchor=north west][inner sep=0.75pt]   [align=left] {$\displaystyle C$};
% Text Node
\draw (393,152) node [anchor=north west][inner sep=0.75pt]   [align=left] {$\displaystyle A$};
% Text Node
\draw (473,152) node [anchor=north west][inner sep=0.75pt]   [align=left] {$\displaystyle Re\ z$};
% Text Node
\draw (370,93) node [anchor=north west][inner sep=0.75pt]   [align=left] {$\displaystyle B$};
% Text Node
\draw (416,66) node [anchor=north west][inner sep=0.75pt]   [align=left] {$\displaystyle D$};


\end{tikzpicture}
	


\end{center}

Since $f$ is continuous at $z$, \\
	$f\left( w \right)  = f\left( z \right)  + \phi\left( \omega \right), \omega \in \text{BD}$\\
where $\lim_{w \to z} \phi(\omega) = 0$\\
Therefore, $$\omega \to z, f(\omega) \to  f \left( z \right) $$ \\

Since $g\left( \omega \right) = \omega$ is a primitive for 1,\\
$\int_{{\text{BD}}}^{{}} {} \: d{\omega} {} = h$\\
Therefore, (to complete proof)
\begin{rem}
\begin{enumerate}
\item The above theorem holds if $f \in  H\left( D \right) $ where $D \subset \mathds{C}$ is any disk. 
\item Let $\Omega \subset \mathds{C}$ be a disk in $\mathds{C}$ and if $f \in H\left( \Omega \right) $ then, \\
	$\int_{{\gamma}}^{{}} {f\left( z \right) } \: d{z} {} = 0$\\
	for any closed contour $\gamma \subset \Omega$
\item If $A \subset \mathds{C}$ and $f$ is holomorphic on $A $ if there is an open set $A \subset U$ such that $f \in  H \left( U \right) $ \\
\item If $\gamma$ is an closed contour then $f $ is said to be analytic on and inside $\gamma$ if $f \in H\left( U \right) $
\end{enumerate}
\end{rem}
\begin{defn}[Simply-connected]
A domain $\Omega \subset \mathds{C}$ is said to be \emph{simply connected} if for every simple closed curve $\gamma$ lying in $\Omega$,   $Int\left( \gamma \right) \subset \Omega$
\end{defn}
Let $f \in H\left( \Omega \right) , \Omega \subset \mathds{C}$ is simply connected and let $\alpha, \beta \in  \Omega$.\\
Then the integral of $f$ along any contour joining $\alpha$	and $\beta$ is same.  
Let $\gamma_1, \gamma_2$ be two distinct contours joining $\alpha, \beta$. Then, consider the path $\gamma = \gamma_1 + -\left( \gamma_2 \right) $. Then, we have that \\
$\int_{{\gamma}}^{{}} {f\left( z \right) } \: d{z} {} = 0$ (By Cauchy-Goursat theorem)\\
\vspace{2mm}
Let $\gamma_1$ and $\gamma_2$ be two simple closed contours with same orientations such that $\left\{ \gamma_2 \right\} \subset Int\left( \gamma_1 \right) $. If a function is holomorphic in the closed contour region bounded by $\gamma_1$ and $\gamma_2$, then, 

	\[
	\int_{{\gamma_1}}^{{}} {f\left( z \right) } \: d{z} {} = \int_{{\gamma_2}}^{{}} {f\left( z \right) } \: d{z} {} 
	.\] 


\begin{center}


\tikzset{every picture/.style={line width=0.75pt}} %set default line width to 0.75pt        

\begin{tikzpicture}[x=0.75pt,y=0.75pt,yscale=-1,xscale=1]
%uncomment if require: \path (0,300); %set diagram left start at 0, and has height of 300

%Shape: Circle [id:dp02448716649342253] 
\draw   (197,145.5) .. controls (197,69.56) and (258.56,8) .. (334.5,8) .. controls (410.44,8) and (472,69.56) .. (472,145.5) .. controls (472,221.44) and (410.44,283) .. (334.5,283) .. controls (258.56,283) and (197,221.44) .. (197,145.5) -- cycle ;
%Shape: Polygon Curved [id:ds8654614936411351] 
\draw   (296,102) .. controls (316,92) and (406,82) .. (386,102) .. controls (366,122) and (366,132) .. (386,162) .. controls (406,192) and (316,192) .. (296,162) .. controls (276,132) and (276,112) .. (296,102) -- cycle ;
%Straight Lines [id:da9010536947679112] 
\draw    (424,42) -- (387.07,100.31) ;
\draw [shift={(386,102)}, rotate = 302.35] [color={rgb, 255:red, 0; green, 0; blue, 0 }  ][line width=0.75]    (10.93,-3.29) .. controls (6.95,-1.4) and (3.31,-0.3) .. (0,0) .. controls (3.31,0.3) and (6.95,1.4) .. (10.93,3.29)   ;
%Shape: Boxed Line [id:dp8382501059640968] 
\draw    (352,185) -- (371.6,282.04) ;
\draw [shift={(372,284)}, rotate = 258.58] [color={rgb, 255:red, 0; green, 0; blue, 0 }  ][line width=0.75]    (10.93,-3.29) .. controls (6.95,-1.4) and (3.31,-0.3) .. (0,0) .. controls (3.31,0.3) and (6.95,1.4) .. (10.93,3.29)   ;
\draw   (355.33,189.67) .. controls (353.5,187.72) and (352.22,185.67) .. (351.48,183.49) .. controls (351.67,185.78) and (351.29,188.17) .. (350.37,190.69) ;
\draw   (422.96,48.08) .. controls (423.32,45.42) and (424.1,43.14) .. (425.33,41.19) .. controls (423.67,42.78) and (421.57,44) .. (419.05,44.87) ;
\draw   (254.25,36.63) .. controls (255.65,34.35) and (257.31,32.58) .. (259.22,31.31) .. controls (257.06,32.08) and (254.65,32.33) .. (251.98,32.1) ;
\draw   (332.82,90.52) .. controls (330.98,92.46) and (329,93.85) .. (326.87,94.7) .. controls (329.14,94.39) and (331.55,94.63) .. (334.11,95.42) ;
\draw   (369.78,136.26) .. controls (371.77,138.06) and (373.21,140) .. (374.12,142.11) .. controls (373.75,139.84) and (373.93,137.43) .. (374.64,134.85) ;

% Text Node
\draw (495,99) node [anchor=north west][inner sep=0.75pt]   [align=left] {$\displaystyle \gamma _{1}$};
% Text Node
\draw (312,188) node [anchor=north west][inner sep=0.75pt]   [align=left] {$\displaystyle \gamma _{2}$};
% Text Node
\draw (352,61) node [anchor=north west][inner sep=0.75pt]   [align=left] {$\displaystyle A$};
% Text Node
\draw (434,22) node [anchor=north west][inner sep=0.75pt]   [align=left] {$\displaystyle B$};
% Text Node
\draw (337.5,159.5) node [anchor=north west][inner sep=0.75pt]   [align=left] {$\displaystyle C$};
% Text Node
\draw (407,271) node [anchor=north west][inner sep=0.75pt]   [align=left] {$\displaystyle D$};
% Text Node
\draw (478,175) node [anchor=north west][inner sep=0.75pt]   [align=left] {$\displaystyle E$};
% Text Node
\draw (176,164) node [anchor=north west][inner sep=0.75pt]   [align=left] {$\displaystyle F$};
% Text Node
\draw (256,113) node [anchor=north west][inner sep=0.75pt]   [align=left] {$\displaystyle G$};
% Text Node
\draw (403,162.4) node [anchor=north west][inner sep=0.75pt]    {$H$};


\end{tikzpicture}	

\end{center}


\begin{restatable}[\textbf{Cauchy's Integral Formula}]{prop}{}\label{}
	Let $\gamma$ be a simple closed contour, and let $z\in  Int\left( \gamma \right) $. If $f$ is analytic on and inside $\gamma$, then, \\
		\[
		f\left( z_0 \right)  = \frac{1}{2 \pi i} \int_{{\gamma}}^{{}} {\frac{f\left( z \right)}{z-z_0} } \: d{z} {}
		.\] 
\end{restatable}
\begin{proof}
Witout loss of generality, let $\gamma$ be  positively orineted. Since $f$ is continuous at $z_0$, $\forall \epsilon >0, \exists \delta >0$ such that, 

	\[
		|z - z_0| < \delta \implies |f\left( z \right) - f\left( z_0 \right)| < \epsilon 
	.\] 

	Set $\gamma_1:= |z-z_0| = r$, where $0<r<\delta$,    \hspace{5mm}(Simple Circle, positively oriented) 

	The function $\frac{f\left( z \right) }{z-z_0}$ is analytic in the closed annular region bounded by $\gamma$ andd $\gamma_1$. Then, by the previous lemma, we have, 

	\begin{equation*}
        \begin{split}
	\int_{{\gamma}}^{{}} {\frac{f\left( z \right) }{z-z_0}} \: d{z} {} = \\
	\int_{{\gamma_1}}^{{}} {\frac{f\left( z \right) }{z-z_0}} \: d{z} {} 
        \end{split}
    \end{equation*}


	Now, we can write the corresponding integral as follows:

	\begin{equation*}
	\int_{{\gamma_1}}^{{}} {\frac{f\left( z \right) }{z-z_0}} \: d{z} {}  = \int_{{\gamma_1}}^{{}} {\frac{f\left( z \right) - f\left( z_0 \right)  }{z-z_0}} \: d{z} {} + \int_{{\gamma_1}}^{{}} {\frac{f\left( z_0 \right) }{z-z_0}} \: d{z} {} 
    \end{equation*}
	Now, 
	\begin{equation*}
		|\int_{{\gamma_1}}^{{}} {\frac{f\left( z \right) - f\left( z_0 \right)  }{z-z_0}} \: d{z} {}| \le  \frac{\epsilon}{r}  (2 \pi r)  
    \end{equation*}

	Hence the first term vanishes as $\epsilon>0$ 

	Now the second term is:

	\[
	\begin{split}
		\int_{{\gamma_1}}^{{}} {\frac{f\left( z_0 \right) }{z-z_0}} \: d{z} {}  &= f\left( z_0 \right) \int_{{\gamma_1}}^{{}} {\frac{1}{z-z_0}} \: d{z} {} \\							&= f\left( z_0 \right) 2 \pi i 
	\end{split}
	.\] 

	Hence, we have that 

	
	\[
	\begin{split}
		\frac{1}{2 \pi i}\int_{{\gamma}}^{{}} {f\left( z \right) } \: d{z} {} &= \int_{{\gamma_1}}^{{}} {f\left( z \right) } \: d{z} {} \\ 
										      &=  \frac{f \left( z_0 \right)}{2 \pi i} 2 \pi i = f\left( z_0 \right)  \\
	\end{split}
	.\] 
\end{proof}
\begin{restatable}[\textbf{Cauchy's Integral Formula for derivatives}]{prop}{}\label{}
Let $\gamma$ be a simple closed contour and let $z_0\in Int(\gamma)$. If $f$ is holomorphic on and inside $\gamma$, then $f$ is infinitely differentiable at any point in $Int(\gamma)$ and\\
\begin{equation*}
    f^{(n)}(z_0)=\frac{n!}{2\pi i}\int_{\gamma}^{}\frac{f(z)}{(z-z_0)^{n+1}} dz
\end{equation*}
\end{restatable}
\begin{proof}
For $n=0$, the conclusion is true (by Cauchy's Integral formula)\\
Suppose it holds for $k=n-1$. Then,\\
\begin{equation*}
    \begin{split}
    &f^{(n-1)}(z_0)=\frac{(n-1)!}{2\pi i}\int_{\gamma}^{}\frac{f(z)}{(z-z_0)^n} dz\\
    &\lim_{h \to 0} \frac{f^{n-1}(z_0+h)-f^{n-1}(z_0)}{h}\\
    &\lim_{h \to 0} \frac{(n-1)!}{2\pi i}\int_{\gamma}^{} f(z).\frac{1}{h}[\frac{1}{(z-z_0-h)^n}-\frac{1}{(z-z_0^n)}] dz\\
    &\lim_{h \to 0} \frac{(n-1)!}{2\pi i}\int_{\gamma}^{} \frac{f(z)}{(z-z_0-h)(z-z_0)}[A^{n-1}+A^{n-2}B+\dots +B^{n-1}] \:\:(\text{where})
    \end{split}
\end{equation*}
(to complete, refer to corollary 4.2 of stein-shakarchi for proof)
\end{proof}
\begin{restatable}[\textbf{Cauchy's estimate}]{prop}{}\label{}
Let $f$ be analytic on and inside a simple circle $\gamma: |z-z_0|=r$ and let $|f(z)|\leq M \:\:\forall z\in \{\gamma\}$. Then,\\
\begin{equation*}
    |f^{(n)}(z_0)|\leq \frac{n!M}{r^n}
\end{equation*}
\end{restatable}
\begin{proof}
\begin{equation*}
    \begin{split}
        |f^{(n)}(z_0)|&\leq \frac{n!}{2\pi}\int_{\gamma}^{} |\frac{f(z)}{(z-z_0)}| dz \\
        & \frac{n!}{2\pi}.\frac{M}{r^{n+1}}2\pi r=\frac{n!M}{r^n}
    \end{split}
\end{equation*}
(to complete, refer to corollary 4.3 of stein-shakarchi for proof)
\end{proof}
\begin{restatable}[\textbf{Liouville's theorem}]{thm}{}\label{}
A function which is entire and bounded is a constant function.
\end{restatable}
\begin{proof}
Let $f$ be an entire function with $|f(z)|\leq M \:\:\forall z\in\mathds{C}$\\
$|f^{'}(z_0)|\leq \frac{M}{R} < \epsilon$, for given $\epsilon>0$ when $R$ is sufficiently large.\\
Therefore, $f^{'}(z_0)=0$
\end{proof}
\begin{restatable}[]{thm}{}\label{}
Let $\Omega\subseteq \mathds{C}$ be a domain and let $f\in H(\Omega)$. If $D$ is a disk centered at $z_0\in \Omega$ such that $\bar{D}\subseteq \Omega$, then $f$ has a power series\\
\begin{equation*}
    \begin{split}
        &f(z)=\sum_{n=0}^{\infty} a_n(z-z_0)^n\text{,}\:\: \forall z\in D\\
        &\text{where}\:\:\: a_n=\frac{f^{(n)}(z_0)}{n!}\text{,} \:\: \forall n=0,1,2,\dots
    \end{split}
\end{equation*}
\end{restatable}
\begin{proof}
Fix $z\in D$\\
\begin{equation*}
 f(z)=\frac{1}{2\pi i}\int_{\gamma}^{}\frac{f(z)}{w-z} dw \:\: \gamma=\partial D
\end{equation*}
\begin{equation*}
    \frac{1}{w-z}=\frac{1}{(w-z_0)-(z-z_0)}=\frac{1}{w-z_0}.\frac{1}{1-\frac{z-z_0}{w-z_0}}
\end{equation*}
$\exists\in (0,1)$ such that \\
\begin{equation*}
    \begin{split}
        &|\frac{z-z_0}{w-z_0}|<r<1
    \end{split}
\end{equation*}
(Refer to Theorem 4.4 of stein-shakarchi)
\end{proof}
%Couldn't complete proof
%Example Questions (2)

\textbf{Questions}:
\begin{enumerate}
\item Prove that every non-constant polynomial with complex coefficients has a root in $\mathds{C}$ \\(A continuous function is bounded inside a disk)\\
\item Every monic polynomial $p(z)$ with degree $n$ ($\geq 1$) has precisely $n$ roots in $\mathds{C}$. If the roots are $z_1,z_2\dots z_n \in \mathds{C}$, then $p(z)= (z-z_1)\dots(z-z_n)$.\\
\end{enumerate}