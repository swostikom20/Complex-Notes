\documentclass{article}
\usepackage[utf8]{inputenc}
\usepackage{a4wide}
\usepackage{dsfont}
\usepackage{amsmath}
\usepackage{upgreek}
\usepackage{setspace} \doublespacing
\usepackage{mathtools}
\usepackage{titlesec}
\titleformat{\section}[block]{\sffamily\Large\filcenter\bfseries}{\S\thesection.}{0.25cm}{\Large}
\titleformat{\subsection}[block]{\large\bfseries\sffamily}{\S\S\thesubsection.}{0.2cm}{\large}
\usepackage{hyperref}
\title{\textbf{MA-412}\\ \textbf{Complex Analysis}}
\author{Om Swostik\\\url{https://github.com/swostikom20/}}
\date{Spring Semester 2022-2023}
\begin{document}
\maketitle
\clearpage
\begin{flushleft}
\section{Preliminaries}
\textbf{Region}: An open, connected subset of $\mathds{C}$ is called domain or region.
\begin{align}
&D(a,r)=\{z\in \mathds{C}|\: |z-a|<r\} \nonumber\\
&\{z\in \mathds{C}|\:r<|z-a|<R\}  \rightarrow \text{annulus} \nonumber
\end{align}
\textbf{Exponential function}: $exp(z):\mathds{C}\rightarrow\mathds{C}$
\begin{align}
&e^z=e^x(cos(y)+isin(y))\nonumber\\
&e^{z_1+z_2}=e^{z_1}.e^{z_2},  0\notin Rng(exp(z))\nonumber\\
&|e^{ix}|=1  \:\: (\forall x\in \mathds{R})\nonumber\\
&e^{z}=1 \:\:\text{iff} \:\: z=2n\pi i\text{,} \:\: n\in \mathds{Z}\nonumber\\
&e^{z_1}=e^{z_2} \:\:\text{iff}\:\: z_1=z_2+2n\pi i\nonumber
\end{align}
\textbf{Argument}: $arg(z):\mathds{C}\setminus \{0\}\rightarrow\mathds{C}$
\begin{align}
&arg(z)=\theta \:\:\text{(Angle made wrt positive real axis)} \nonumber\\
&arg(z) \:\:\text{for} \:\: z\in \mathds{C}\setminus \{0\} \:\:\text{is multivalued function} \nonumber\\
&arg(z): \mathds{C}\setminus \{0\}\rightarrow (\alpha,\alpha+2\pi] \:\:\text{or}\:\: [\alpha,\alpha+2\pi) \:\:\text{is well-defined}\:\: (\forall \alpha \in \mathds{R})\nonumber
\end{align} 
\textbf{Principal value}: 
$\begin{aligned}[t] 
&Arg(z)=\theta  \:\: (-\pi<\theta\leq \pi)\\ &arg(z)=\{Arg(z)+2n\pi:n\in \mathds{Z}\}
\end{aligned}$ \:\:
\\
\vspace{5mm}
\textbf{Complex Log}: For $z\in \mathds{C}\setminus \{0\}$, define: \begin{equation} log(z)=log(|z|)+ iarg(z) \nonumber \end{equation} 
Notice that $log$ is multivalued\\
\clearpage
Whenever $arg(z)$ is well-defined (i.e $Rng(arg(z))=[\alpha,\alpha+2\pi)$ or $(\alpha,\alpha+2\pi]$)\\$\Rightarrow log(z)$ is well defined
\begin{align}
&Log(z)=log(|z|)+i.Arg(z) \:\:\text{(principal log)} \nonumber\\
&log(z)=Log(z)+2n\pi i \nonumber
\end{align}
\textbf{Lemma 1.1}: For $z\in \mathds{C}\setminus \{0\}$, the values of $log(z)$ are the complex numbers $\omega$ such that $e^{\omega}=z$\\
\textbf{Proof}: Do it on your own!\\
\vspace{3mm}
\textbf{Limit}: $f: E\subset\mathds{C} \rightarrow \mathds{C}$
\begin{align}
&\lim_{z \to z_0} f(z)=l \nonumber\\
&\forall \epsilon>0,\exists \delta>0 \:\:\text{such that}\:\: 0<|z-z_0|<\delta \Rightarrow |f(z)-l|<\epsilon\nonumber\\
&z \in (B(z,z_0)\setminus\{0\})\cap E \Rightarrow f(z)\in B(l,\epsilon) \nonumber
\end{align}
\textbf{Proposition 1.1 (Sequential criteria for limit)}: \\ A function $f:E\subset  \mathds{C}\rightarrow \mathds{C}$ has a limit point $l$ as $z\rightarrow z_0$ iff $f(z_n)\rightarrow l$ for every sequence $\{z_n\}\subset E\setminus \{z_0\}$ with $z_n\rightarrow z_0$ as $n\rightarrow \infty$\\
\vspace{2mm}
\textbf{Theorem 1.1}: If $f:E\subset \mathds{C} \rightarrow \mathds{C}$ has a limit at $z_0$, then $f$ is bounded near $z_0$\\
\vspace{2mm}
\textbf{Continuity}: $f:E\subset \mathds{C}\rightarrow \mathds{C}$\\
$f$ is continuous at $z_0\in E$ if for every basic neighbourhood $V$ of $f(z_0)$, there is a basic neighbourhood $U$ of $z_0$ such that $f(U)\subset V$. Let $V=B(f(z_0),\epsilon)$ \& $U=B(z_0,\delta)$,
\begin{center}
$z\in B(z_0,\delta)\Rightarrow f(z)\in B(f(z_0),\epsilon)$\\
$|z-z_0|<\delta \Rightarrow |f(z)-f(z_0)|<\epsilon$\\
Then, $z\in B(z_0,\delta)\setminus \{z_0\} \Rightarrow f(z)\in B(f(z_0),\epsilon)$\\
$\lim_{z \to z_0} f(z)=f(z_0)$
\end{center}
\textbf{Prove}: Let $f:G\rightarrow \mathds{C}$ such that $f=u+ iv$ (where $u,v:G\rightarrow \mathds{R}$)

$f$ is continuous at a point $z_0$ iff $u,v$ are continuous at $z_0$

\clearpage

\section{Branch of Log and Power function}
$Arg: \mathds{C}\setminus \{0\} \rightarrow (-\pi,\pi]$\\
$Log(z)=log(|z|)+i.Arg(z)$ (defined on $\mathds{C}\setminus \{0\}$)\\
$Log(z)$ is continuous on $\mathds{C}\setminus (-\infty,0]$ as $Arg(z)$ is continuous on $\mathds{C}\setminus (-\infty,0]$\\
\subsection{\textbf{Branch of Log}}
Let $G\subset \mathds{C}$ be a region. A continuous function $f:G\rightarrow \mathds{C}$ is called a branch of logarithm in $G$ if:
\begin{align}
&e^{f(z)}=z \:\:(\forall z\in G) (\Rightarrow 0\notin G)\nonumber\\
&e^{Log(z)}=z \:\: (\forall z\in \mathds{C}\setminus \{0\})\nonumber
\end{align}
But $Log(z)$ isn't continuous on $\mathds{C}\setminus\{0\}$\\
Therefore, $Log(z)$ is a branch of log in $\mathds{C}\setminus (-\infty,0]$ 
\begin{align}
&f_k=Log(z)+2k\pi i\text{,}\:\: k\in \mathds{Z}\nonumber \\
&e^{f_k(z)}=z \:\:(\text{Each $f_k$ is a branch of log in G})\nonumber
\end{align}
\textbf{Theorem 2.1}: Let $f:G\rightarrow \mathds{C}$ be a branch of log. Then $g:G\rightarrow \mathds{C}$ is a branch of log iff $g(z)=f(z)+2k\pi i$ (for some $k\in \mathds{Z}$)

\textbf{Proof}: ($\Rightarrow$) if $f$ is a branch of log, then so is $g$

($\Leftarrow$) Let $g:G\rightarrow \mathds{C}$ be a branch of log. 

Then, (i) $g$ is continuous

(ii) $e^{g(z)}=z$, $\forall z\in G$

Since $f:G\rightarrow \mathds{C}$ is a branch of log, we have, 

(i) $f$ is continuous 

(ii)$e^{f(z)}=z$, $\forall z\in G$

$e^{g(z)}=z=e^{f(z)}$ ($\forall z\in G$)

$\Rightarrow g(z)=f(z)+ 2k(z)\pi i$ ($k$ depends on $z$)

$k: G\subset \mathds{C}\rightarrow \mathds{Z}$

$k(z)=\frac{1}{2\pi i}(g(z)-f(z))$ ($k$ is continuous)

Since $G$ is connected, $Img(k)$ is connected  (subset of $\mathds{Z}$)

$\Rightarrow k(z)$ is constant 

Hence, the claim follows.

\textbf{Unit disk in $\mathds{C}$}: $D=\{z\in \mathds{C}: |z|< 1\}$

$0\in D$, hence $D$ cannot be a branch of log. 

$D\setminus \{0\}$ isn't a branch of log. (Why?)

$Log(z): \mathds{C}\setminus (-\infty,0]\rightarrow \mathds{C}$

The half-line is called a branch-cut for any member of $\{Log(z)+2k\pi i: k\in \mathds{Z}\}$

\textbf{Question}: How do you make a branch-cut to define a branch of $Log(z+i-1)$?

\subsection{\textbf{Power functions}}

Let $\alpha \in \mathds{C}$. We define $z^{\alpha}$ to be the multi-valued function:
\begin{align} 
z^{\alpha}=e^{\alpha log(z)}&= e^{\alpha(log(|z|)+ iarg(z))}\nonumber\\
&= e^{\alpha(Log(z)+2k\pi i)}\nonumber\\
&=e^{\alpha Log(z)}.e^{2\pi ik\alpha} \:\:(z\neq 0\text{,} \:k\in \mathds{Z})\nonumber
\end{align}
Let $\alpha=n \in \mathds{N}$. Then $z^{\alpha}=e^{nLog(z)}$ which is single-valued.
\clearpage
\section{Differentiability and Power series}

\subsection{\textbf{Differentiability}} 

Let $\Omega \subset \mathds{C}$ and $f: \Omega \rightarrow \mathds{C}$. Then $f$ is said to be differentiable at $z_0\in \Omega$ if,

$$f^{'}(z_0)= \lim_{z \to z_0} \frac{f(z)-f(z_0)}{z-z_0}$$ exists 

Example: $f(z)=\bar{z}$ isn't differentiable at any point in $\mathds{C}$.

\textbf{Cauchy-Riemann equations}: Let $f:\Omega \rightarrow C$ be differentiable at $z_0=x_0+iy_0$. Then,

$$\frac{\partial u}{\partial x}(x_0,y_0)=\frac{\partial v}{\partial y}(x_0,y_0)$$

$$\frac{\partial u}{\partial y}(x_0,y_0)= -\frac{\partial v}{\partial x}(x_0,y_0)$$

Further, $$f^{'}(z_0)= \frac{\partial u}{\partial x}(x_0,y_0)+i\frac{\partial v}{\partial x}(x_0,y_0)$$

Note that the last equation only has partial derivatives wrt $x$.

\textbf{Proof (Sketch)}: In the differential, approach the origin once through the real axis and and then the imaginary axis to get some equations  

\textbf{Holomorphic/Analytic}: A function is said to be \textbf{holomorphic} or \textbf{analytic} at $z_0$ if it is differentiable in a neighbourhood of $z_0$. 

If a function is analytic at a point, it is called a regular point for the function 

\textbf{Entire functions}: If a function $f$ is analytic at every point in $\mathds{C}\Rightarrow f$ is entire.

Example: 
\begin{enumerate}
\item $f(z)\rightarrow$ polynomial in $\mathds{C}$ 
\item$f(z)=\frac{1}{z}$ (differentiable at every point on $\mathds{C}\setminus \{0\}$)
\end{enumerate}
\textbf{Proposition 3.1}: If $f$ and $g$ are differentiable at $z_0\in \Omega \subset \mathds{C}$, then so are:

(i)$f+g$, (ii)$fg$, (iii)$f/g$ ($g(z_0)\neq 0$)

\textbf{Proof}: Trivial

\textbf{Chain Rule}: $f:\Omega \rightarrow U$, $g:U\rightarrow \mathds{C}$ are holomorphic at $z_0$ and $f(z_0)$ respectively.
Then $g\circ f:\Omega\rightarrow \mathds{C}$ is holomorphic at $z_0$ and $(g\circ f)^{'}(z_0)=g^{'}(f(z_0)).f^{'}(z_0)$

\textbf{Proof}: Standard (found in any introductory complex analysis text)

\subsection{\textbf{Power series}} 

A series of the form $\sum_{0}^{\infty} a_n(z-z_0)^n$ ($a_n,z,z_0\in \mathds{C}$; $z_0$ is center)

\textbf{Theorem 3.1(Radius of convergence)}: Given a power series $\sum_{n=0}^{\infty} a_n(z)^n$,\\
$\exists R\in [0,\infty) \cup \{\infty\}$ such that:
\begin{enumerate}
\item the series converges absolutely ($\forall z\in B(0,R)$; the disk of convergence)
\item diverges $\forall z$ such that $|z|>R$ 
\item \begin{equation} R=\frac{1}{\limsup |a_n|^{\frac{1}{n}}}\nonumber \end{equation}
\end{enumerate}
(Convention: $\frac{1}{0}=+\infty$ , $\frac{1}{\infty}=0$)\\
\textbf{Proof}: Can be found in any standard text (Try Conway!)\\
\vspace{2mm}
\textbf{Theorem 3.2}: The power series $f(z)=\sum_{n=0}^{\infty} a_nz^n$ defines a holomorphic function (in its disk of convergence)\\
Derivative of $f$ is obtained by differentiation of each term:\\
$f^{'}(z)=\sum_{n=0}^{\infty} na_nz^{n-1}$ \\
$f^{'}$ has same disk of convergence\\
\textbf{Proof}: Found in any standard text\\
\textbf{Remark}: A power series is infinitely differentiable (analytic) in its disk of convergence\\  
$\Rightarrow$ \hspace{40mm}$ \begin{aligned} f(z)=\sum_{}^{}a_nz^n =\sum_{k=0}^{\infty} \frac{f^{k}(0)}{k!} z^k \nonumber\\ \end{aligned}$
\\
\vspace{3mm}
\textbf{Proposition 3.2}: Let $\Omega_1,\Omega_2 \subset \mathds{C}$ be domains and let $f:\Omega_1\rightarrow \mathds{C}$,
$g:\Omega_2\rightarrow \mathds{C}$ be continuous function such that $f(\Omega_1)\subset \Omega_2$ and that $g(f(z))=z$ ($\forall z\in \Omega_1$).
If $g$ is differentiable on $\Omega_2$ and if $g^{'}(w)\neq 0$ ($\forall w \in \Omega_2$), then $f$ is differentiable and $f^{'}(z)=\frac{1}{g^{'}(f(z))}$ ($\forall z\in \Omega_1$)

\textbf{Proof (Sketch)}: Use the differentiability of $g$ and the continuity of $f$ to get the result \\
\vspace{3mm}
\textbf{Theorem 3.3}: A branch of logarithm is analytic and its derivative is $\frac{1}{z}$.

\textbf{Proof (Sketch)}: Let $f$ be a branch of log. Define $g(z)=e^z$. Then $g(f(z))=z$. ($e^{f(z)}=z$)
\clearpage
\textbf{Lemma 3.1}: Let $f:\Omega \rightarrow \mathds{C}$. Then $f$ is differentiable at $z_0\in \Omega$ iff $\exists a\in \mathds{C}$ such that 

$f(z_0+h)-f(z_0)= ah+h\psi(h)$ ($\lim_{h \to 0} \psi(h)=0$)

\textbf{Proof}: Use the definition of differentiability of a multivariable function\\
\vspace{3mm}
\textbf{Theorem 3.4}: Let $\Omega \subset \mathds{C}$ be a domain and let $f=u+iv$ is a function from $\Omega$ to $\mathds{C}$.
Then $f$ is analytic at $z_0\in \Omega$ iff the partial derivatives exist, are continuous and satisfy Cauchy-Riemann equations.

\textbf{Proof(To complete)}: $\Rightarrow$ If $f$ is analytic at $z_0$, then partial derivatives are continuous and satisfy the Cauchy-Riemann equations (follows from theorem 3.2)

$\Leftarrow$ Let $\tilde{f}:\Omega \subseteq \mathds{R}^2\rightarrow \mathds{R}^2$ be defined as $\tilde{f}=(u,v)$

$\tilde{f}$ is differentiable at $z_0=(x_0,y_0)$\\
\vspace{3mm}
\textbf{Definition}: For domain $\Omega$, $H(\Omega)$ is the collection of all analytic functions from $\Omega \rightarrow \mathds{C}$

\textbf{Proposition 3.3}: Let $f:\Omega\subseteq \mathds{C} \rightarrow \mathds{C}$ such that $f\in H(\Omega)$. If $D$ is a disk with center at $z_0\in \Omega$ and if $\bar{D}\subseteq \Omega$, then $f$ has a power series:
\begin{align}
f(z)=\sum_{n=0}^{\infty} a_n(z-z_0)^n \:\:(\forall z\in D) \nonumber\\
\text{where}\: a_n=\frac{f^{(n)}(z_0)}{n!}\text{,}\:\: (n\in \mathds{N})\nonumber
\end{align}
\textbf{Proof (Sketch)}: Differentiate the power series repeatedly (Use Theorem 3.2)

\textbf{Examples}: 
\begin{enumerate}
\item $\begin{aligned}[t] \sum_{n=1}^{\infty} nz^n <\infty \end{aligned}$ \: (converges $\forall z \in B(0,1)$; doesn't converge for $|z|=1$) 
\item $\begin{aligned}[t] \sum_{n=1}^{\infty} \frac{z^n}{n^2} <\infty \end{aligned}$ \:(converges $\forall z\in B(0,1)$; converges for $|z|=1$)
\item $\begin{aligned}[t] \sum_{n=1}^{\infty} \frac{z^n}{n} <\infty \end{aligned}$ \:(converges for $|z|<1$; diverges otherwise)
\item $f(z)=e^z \Rightarrow f^{'}(z)=e^z$ (by definition of $e^z$)\\
Consider $\begin{aligned}[t] g(z)=\sum_{n=1}^{\infty} \frac{z^n}{n!}\end{aligned}$ \\
$g(z)$ is an entire function (Why?)\\
Note that $\begin{aligned}[t] f^{(k)}(0)=1\Rightarrow a_k=\frac{f^{(k)}(0)}{k!}=\frac{1}{k!}\Rightarrow f(z)=g(z) \end{aligned}$
\end{enumerate}
\clearpage

\section{Complex integration}

A \textbf{path} or a \textbf{curve} is a continuous function, $\gamma:[a,b]\rightarrow \mathds{C}$ 
($Rng(\gamma)\subset \mathds{C}$) 

$\gamma(a)$: initial point of path; $\gamma(b)$: endpoint of path

$[a,b]$: parameter interval 

$\gamma$ is said to be:
\begin{enumerate}
    \item closed if $\gamma(a)=\gamma(b)$
    \item smooth or $C^1$ if $\gamma$ is differentiable and $\gamma^{'}$ is continuous
    \item simple if $\gamma$ is one-one
    \item simple closed if $\gamma(a)=\gamma(b)$ and $\gamma$ is one-one on $(a,b)$
    \item piecewise smooth if there are finitely many points $s_0,s_1 \dots s_n\in [a.b]$ with $a=s_0<s_1<s_2 \dots <s_n=b$ such that the restriction of $\gamma$ to each $(s_i,s_{i+1})$ is smooth.

\end{enumerate}

$-\gamma$ or $\gamma^{-1}$ is defined by $\gamma^{-1}(t)=\gamma(a+b-t)$

$\upphi:[0,1]\rightarrow [a,b]$ defined as: $\upphi(t)=a+(b-a)t$ (one-one and differentiable)

\subsection{\textbf{Line integral}} 

$f:[a,b]\rightarrow \mathds{C}$ : continuous

$f=u+iv$, where $u,v:[a,b]\rightarrow \mathds{R}$

Define $\int_{a}^{b}f(t) dt= \int_{a}^{b}u(t) dt+ i \int_{a}^{b} v(t) dt$

\textbf{Properties}:

\begin{enumerate}
    \item $\int_{a}^{b} c.f(t) dt= c.\int_{a}^{b}f(t) dt$
    \item $| \int_{a}^{b}f(t) dt |\leq \int_{a}^{b}|f(t)| dt$
\end{enumerate}
\textbf{Length of a smooth curve}: Let $\gamma:[a,b]\rightarrow \mathds{C}$ be a smooth curve. 

$L(\gamma)= \int_{a}^{b} |\gamma^{'}(t)| dt$
$=\int_{a}^{b} \sqrt{\gamma_1^{'}(t)^2+\gamma_2^{'}(t)^2}  dt$     \: ($\gamma(t)=\gamma_1(t)+i.\gamma_2(t)$)

If $\gamma:[a,b]\rightarrow C$ is piecewise smooth then $L(\gamma)$ is the sum of the length of its smooth parts.

\textbf{Orientation}:  A curve $\gamma$ is \emph{positively} oriented if traversed in anti-clockwise direction else is \emph{negatively} oriented.

\textbf{Examples}:
\begin{enumerate}
    \item $\gamma(t)=re^{it}$, ($t\in [0,2\pi]$) ($r>0$: simple, smooth curve); 
    
    $L(\gamma)=\int_{0}^{2\pi} |ire^{it}| dt= r(2\pi)$
    
    \item $\gamma(t)=e^{it}$, ($t \in [0,4\pi]$): closed, smooth, traverses the unit circle twice in the positive direction
\end{enumerate}
\textbf{Integration over paths}:
$\gamma[a,b]\rightarrow \mathds{C}$ is a smooth curve and $f:\gamma \rightarrow \mathds{C}$: continuous 

\textbf{Definiton}: $\int_{\gamma}^{} f(z) dz= \int_{a}^{b} f(\gamma(t)) \gamma^{'}(t) dt=\int_{a}^{b}g(t) dt$ 

($g(t)=f(\gamma(t)) \gamma^{'}(t)$ where $g:[a,b]\rightarrow \mathds{C}$)

Let $[a_1,b_1]$ be any closed interval. Then $\exists \upphi:[a_1,b_1]\rightarrow [a,b]$ (one-one,differentiable and $\upphi(a_1)=a;\upphi(a_2)=b$)

$\upphi[a_1,b_1] \rightarrow \mathds{C}$: smooth 

$\int_{a_1}^{b_1}f(\gamma_1(t)).\gamma_1^{'}(t)dt$ \: ($=\int_{\gamma_1}^{}f(z) dz$)

$=\int_{a_1}^{b_1}f(\gamma(\upphi(t))).\upphi^{'}(t) dt$
$=\int_{\gamma}^{} f(\gamma(s)) \gamma^{'}(s) ds=\int_{\gamma}^{}f(z) dz$ \: ($\upphi(t)=s$)

If $\gamma$ is piecewise smooth, the integral can be split into the sum of its smooth components:

if $\gamma=\gamma_1+\gamma_2 \dots +\gamma_n$, then $\int_{\gamma}^{} f = \int_{\gamma_1}^{} f + \dots + \int_{\gamma_n}^{} f$.

Note that $\gamma_i's$ are smooth.

\textbf{Proposition}: If $f$ and $g$ are continuous on a smooth curve $\gamma$, then 
\begin{enumerate}
    \item $\int_{\gamma}^{} \alpha f+\beta g = \alpha \int_{\gamma}^{} f + \beta\int_{\gamma}^{} g$
    \item $\int_{\gamma^{-}}^{} f = - \int_{\gamma}^{} f$
    \item $|\int_{\gamma}^{} f(z) dz| \leq \|f\|_{\infty,\gamma} \: L(\gamma)$ \: ($\|f\|_{\infty,\gamma}=sup_{z\in \{\gamma\}}|f(z)|$)
\end{enumerate}

$|\int_{\gamma}^{} f|= |\int_{a}^{b} f(\gamma(t)).\gamma^{'}(t) dt| \leq \int_{\gamma}^{} |f(\gamma(t)).\gamma^{'}(t)| dt$
$\leq \|f\|_{\infty,\gamma}\:  \int_{a}^{b} |\gamma^{'}(t)| dt$   \: ($L(\gamma)=\int_{a}^{b} |\gamma^{'}(t)| dt$)
\textbf{Examples}: 

(i)Let $\gamma$ be the arc of a circle of radius 3 ($|z|=3$) from $3$ to $3i$.

Show that:

$$|\int_{\gamma}^{} \frac{z+4}{z^3-1} dz| \leq \frac{21\pi}{52}$$

(ii) $\gamma: |z|=2$  (traverse curve in positive direction)

Prove: $$|\int_{\gamma}^{} \frac{e^z dz}{z^2+1} | \leq \frac{4\pi e^2}{3}$$
\clearpage
\textbf{Fundamental theorem of calculus}: 

If $f:[a,b]\rightarrow \mathds{R}$ has a primitive F, then $\int_{a}^{b} f(x) dx =F(b)-F(a)$ \: ($F^{'}(x)=f(x), \forall x\in [a,b]$)

\textbf{For complex case}: Suppose $G\in \mathds{C}$ be a domain. If a continuous function $f:G\rightarrow \mathds{C}$ has a primitive $F$ on $G$ and if $\gamma$ is a smooth curve in G with initial and terminal points $\omega_1$ and $\omega_2$ respectively, then:

$\int_{\gamma}^{} f = F(\omega_1) - F(\omega_2)$

\textbf{Proof}: Let $[a,b]\in \mathds{R}$ be a parameter interval for $\gamma$ and $\gamma(a)=\omega_1$; $\gamma(b)=\omega_2$

Given $F^{'}(z)=f(z)$   ($\forall z \in G$)

$$\int_{\gamma}^{} f = \int_{a}^{b} f(\gamma(t)).\gamma^{'}(t) dt = \int_{a}^{b} F^{'}(\gamma(t))\gamma^{'}(t) dt$$

$$= \int_{a}^{b} (F \circ \gamma)^{'}(t) dt = F \circ \gamma(b)-F \circ \gamma(a)= F(\omega_2)-F(\omega_1)$$

\textbf{Corollary-1}: If $\gamma$ is a closed curve (smooth), then

$\begin{aligned}[t] \int_{\gamma}^{} f =0\nonumber \end{aligned}$ 

\textbf{Proof}: Follows from FTC

\textbf{Corollary-2}: If $f\in H(\Omega)$ for a region $\Omega\in \mathds{C}$ and if $f^{'}=0$ on $\Omega$, then $f$ is a constant function.

\textbf{Proof}: Fix a point $\omega_0\in \Omega$. It suffices to show that $f(\omega)=f(\omega_0), \forall \omega \in \Omega$

\subsection{\textbf{Simple Closed Curve}}

\textbf{Jordan-curve theorem}: Every simple closed curve in $\mathds{C}$ divides the complex plane into two regions. One of these regions is bounded and the other is unbounded. The bounded region is called the interior of the curve.

\textbf{Example}: $G=\mathds{C}\setminus \{0\}$

$f(z)=\frac{1}{z}$ on $G$, $\gamma: |z|=1$, $\gamma(t)=e^{it}$, ($t\in [0,2\pi]$)

$$\int_{\gamma}^{} f= \int_{0}^{2\pi} f(\gamma(t)).\gamma^{'}(t) dt= \int_{0}^{2\pi} \frac{i.e^{it}}{e^{it}} dt =2\pi i\neq 0$$

\textbf{Winding number or index of a closed curve}:
Let $\gamma$ be a closed curve on $\mathds{C}$ and let $\alpha \in \mathds{C}\setminus \{\gamma\}$. The winding number of $\gamma$ about $\alpha$ or the index of $\gamma$ with respect to $\alpha$ is denoted by,
$\eta(\gamma;\alpha)/Ind_{\gamma}(\alpha)$ defined by:

$$\eta(\gamma;\alpha)=\frac{1}{2\pi i}\int_{\gamma}^{}\frac{dz}{z-\alpha}$$

\textbf{Example}: $\gamma: [0,6\pi]\rightarrow \mathds{C}$ 

$\gamma(t)=a+re^{it}$ 

$$\eta(\gamma;\alpha)=\frac{1}{2\pi i}\int_{\gamma}^{} \frac{\gamma^{'}(t)}{\gamma(t)-a}=\frac{1}{2\pi i}\int_{\gamma}^{} \frac{1}{a+re^{it}-a}.ire^{it} dt=3$$

\textbf{Theorem 4.1}: Let $\gamma$ be a smooth, closed curve in $\mathds{C}$. Let $\alpha\in \mathds{C}\setminus \{\gamma\}$. Then $\eta(\gamma;\alpha)\in \mathds{Z}$.

\textbf{Proof}: $\upphi:[0,1] \rightarrow \mathds{C}$

$\begin{aligned}[t] \upphi= \frac{\gamma^{'}(s)}{\gamma(s)-\alpha} \end{aligned}$ and $g:[0,1]\rightarrow \mathds{C}$, $\begin{aligned}[t]g(t)=\int_{0}^{t} \upphi(s) ds\end{aligned}$

$g(0)=0$ and $\begin{aligned}[t] g(1)=\int_{0}^{1} \upphi(s) ds= \int_{0}^{1} \frac{\gamma^{'}(s)}{\gamma-\alpha} ds= \int_{\gamma}^{} \frac{dz}{z-\alpha} \end{aligned}$\\
\vspace{2mm}
\textbf{Claim}: $\begin{aligned}[t] g^{'}(t)=\upphi(t) \end{aligned}$

Proof: To show that $\begin{aligned}[t] \lim_{h \to 0}\frac{g(t+h)-g(t)}{h}-\upphi(t)=0 \end{aligned}$

$\begin{aligned}[t] \frac{g(t+h)-g(t)}{h}-\upphi(t)=\frac{1}{h}\int_{t}^{t+h}[\upphi(t+h)-\upphi(t)] ds \end{aligned}$ \:(for $h>0$, similar for $h<0$)

Since $\upphi$ is uniformly continuous on $[0,1]$, $\forall \epsilon >0$, $\exists \delta>0$ such that 

$|s-t|<\delta \Rightarrow |\upphi(s)-\upphi(t)|<\epsilon$ 

If $h<\delta$, then

$\begin{aligned}[t] |\frac{1}{h}\int_{t}^{t+h}[\upphi(t+h)-\upphi(t)]ds|\leq \frac{1}{h}\int_{t}^{t+h}|\upphi(t+h)-\upphi(t)| ds<\epsilon \end{aligned}$

Same thing holds if $h<0$

Therefore, $\begin{aligned}[t] h<\delta \Rightarrow |\frac{g(t+h)-g(t)}{h}-\upphi(t)|<\epsilon \end{aligned}$

Hence, $g^{'}=\upphi$

Set $\begin{aligned}[t] h(t)=e^{-g(t)}(\gamma(t)-\alpha)\end{aligned}$

We have, $\begin{aligned}[t] h^{'}(t)=e^{-g(t)}\gamma^{'}(t)-e^{-g(t)}(\gamma(t)-\alpha)g^{'}(t)=0\end{aligned}$

Hence, $h(t)$ is a constant function.
\begin{align}
&e^{-g(0)}(\gamma(0)-\alpha)=e^{-g(1)}\:\:(\gamma(1)-\alpha)\nonumber \\
\Rightarrow &e^{-g(0)}=e^{-g(1)}=1 \:\:(\text{As $\gamma$ is a closed curve})\nonumber
\end{align}
hence $g(1)=2k\pi i$ (for $k\in \mathds{Z}$)

\clearpage

Therefore, $\begin{aligned}[t] \int_{\gamma}^{}\frac{dz}{z-\alpha}=2k\pi i\Rightarrow \frac{1}{2\pi i}\int_{\gamma}^{}\frac{dz}{z-\alpha}=k\in \mathds{Z} \end{aligned}$ 

Remark: The theorem is true if $\gamma$ is a closed contour. (Prove it!) \\(A contour is a piecewise smooth curve)\\
\vspace{3mm}
\textbf{Theorem 4.2}: Let $\gamma$ be a closed contour and let $\alpha \in \mathds{C}\setminus \{\gamma\}$. Then,

(a) the function $f_{\gamma}: \mathds{C}\setminus \{\gamma\}\rightarrow \mathds{Z}$ is continuous. ($\alpha \rightarrow \eta(\gamma;\alpha)$)

(b) $f$ is constant on every component of $\mathds{C}\setminus \{\gamma\}$

\textbf{Proof}: (a) Let $\alpha_0 \in \mathds{C}\setminus \{\gamma\}$. Then the function $g:t \rightarrow |\alpha_0-\gamma(t)|$ is continuous.

$g$ attains its infimum, say $\begin{aligned}[t] s=\inf_{t\in[0,1]} g(t)\end{aligned}$

If $\alpha$ is very close to $\alpha_0$, then $\begin{aligned}[t]|\alpha-\gamma(t)|\geq \frac{s}{2}\end{aligned}$. Then,
\begin{align}
&|\frac{1}{z-\alpha}-\frac{1}{z-\alpha_0}|=\frac{|\alpha-\alpha_0|}{|z-\alpha||z-\alpha_0|}\leq \frac{2}{s^2}|\alpha-\alpha_0| \:\:(z\in \gamma)\nonumber\\
&|f_{\gamma}(\alpha)-f_{\gamma}(\alpha_0)|\leq \frac{1}{2\pi i} \int_{\gamma}^{}|\frac{1}{z-\alpha}-\frac{1}{z-\alpha_0}| dz\nonumber\\
&\hspace{25mm}\leq \frac{2}{s^2}|\alpha -\alpha_0|\frac{1}{2\pi i}\:L(\gamma)=M(\alpha-\alpha_0) \:\: (\text{Lipschitz continuous $\Rightarrow$ continuous})\nonumber
\end{align}
(b) Let $V$ be a component, then $f(V)$ is connected in $\mathds{Z}\Rightarrow f(V)$ is a constant $\in \mathds{Z}$

\textbf{Proposition 4.1}: Let $\gamma$ be a closed contour in $\mathds{C}$. Then $\eta(\gamma;\alpha)=0$ $\forall \alpha$ in the unbounded component of $\mathds{C}\setminus \{\gamma\}$

\textbf{Proof}: Since $\gamma$ is closed and bounded, $\{\gamma\}\subseteq \bar{B}(0;R)$ for some $R>0$.

Let $\alpha \in\mathds{C}\setminus \bar{B}(0;R)$ 

$|z-\alpha|\geq |\alpha|-|z|\geq |\alpha-R|$

$\begin{aligned}[t] |\eta(\gamma;\alpha)|=\frac{1}{2\pi}|\int_{\gamma}^{}\frac{dz}{z-\alpha}| \leq \frac{1}{2\pi} \int_{\gamma}^{}\frac{dz}{|z-\alpha|}\leq \frac{1}{2\pi} \frac{1}{|\alpha|-R}\:L(\gamma)\end{aligned}$

One can find a large enough $|\alpha|$ to make $\eta(\gamma;\alpha)<1$

Hence, $\eta(\gamma;\alpha)=0$, when $|\alpha|$ is sufficiently large 

Since, $\eta(\gamma;\alpha)$ is constant within a component, $\eta(\gamma,\beta)=0 \:\:(\forall \beta$ in unbounded component)

\textbf{Proposition 4.2}: Let $\gamma$ be a closed contour consisting of curves $\gamma_1,\dots \gamma_n$.Then,

$\eta(\gamma;\alpha)=\eta(\gamma_1;\alpha)+\dots +\eta(\gamma_n;\alpha)$ (Prove!)

\textbf{Cauchy-Goursat theorem}: Let $\Omega\subseteq \mathds{C}$ be a domain and let $f\in H(\Omega)$. Then for any closed contour $\gamma$ lying in the interior of $\Omega$,

$\int_{\gamma}^{} f(z) dz=0$

\textbf{Proof}: Step-I (Goursat's theorem): 

When $\gamma=T$, a triangle

Let $T^{(0)}=T$

Let $diam(T^{(0)})=d^{(0)}$ and $peri(T^{(0)})=p^{(0)}$

%figure

$\int_{T^{(0)}}^{} f(z) dz =\int_{T^{(1)}}^{} f(z) dz + \int_{T^{(2)}}^{} f(z) dz + \int_{T^{(3)}}^{} f(z) dz + \int_{T^{(4)}}^{} f(z) dz$

$|\int_{T^{(0)}}^{} f(z) dz| \leq 4|\int_{T^{(j)}}^{} f(z) dz|$ (for some $j\in \{1,2,3,4\}$)

Call this $T^{(j)}$ to be $T^{(1)}$ (suppose)

$diam(T^{(1)})=\frac{1}{2}diam(T^{(0)})$ 

$d^{(1)}=\frac{d^{(0)}}{2}$ and $p^{(1)}=\frac{p^{(0)}}{2}$

Do the same process with $T^{(1)}$ to get $T^{(2)}\Rightarrow |\int_{T^{(1)}}^{} f(z) dz| \leq 4|\int_{T^{(2)}}^{} f(z) dz|$ 

Continuing, $|\int_{T^{(0)}}^{} f(z) dz| \leq 4^n|\int_{T^{(n)}}^{} f(z) dz|$ 

$d^{(n)}=\frac{d^{(0)}}{2^n}$ and $p^{(n)}=\frac{p^{(0)}}{2^n}$

$\triangle_n= T^{(n)} \cup Int(T^{(n)})$ (Int refers to interior of triangle)

$\triangle_0\supseteq \triangle_1 \supseteq \dots \triangle_n\supseteq \dots$ (nested compact sets)

$d^{(n)}$ tends to $0$

Therefore, $\exists!z_0 \in \overset{\infty}{\underset{n=0}{\bigcap}} \triangle_n$

$f$ is holomorphic at $z_0$

$f(z_0+h)-f(z_0)=hf^{'}(z_0)+h\psi(h)$ ($\lim_{h \to 0}\psi(h)=0$)

So, $f(z)-f(z_0)=(z-z_0)f^{'}(z_0) + (z-z_0)\psi_1(h)$ where $\lim_{z \to z_0}\psi_1(z)=0$

$\Rightarrow \int_{T}^{} f(z) dz= \int_{T}^{} f(z_0) dz + \int_{T}^{} (z-z_0)f^{'}(z_0) dz + \int_{T}^{} (z-z_0)\psi_1(z) dz= \int_{T}^{} (z-z_0)\psi_1(z) dz$

Then, $\psi_1(z)= \frac{f(z)-f(z_0)}{z-z_0}-f^{'}(z_0)$

Let $\sup_{z\in T^{(n)}}|\psi_1(z)|= E_n$ ($E_n\rightarrow 0$ as$\rightarrow \infty$)

$|\int_{T^{(n)}}^{} f(z) dz|=|\int_{T^{(n)}}^{} (z-z_0)\psi_1(z) dz|\leq \int_{T^{(n)}}^{} |z-z_0||\psi_1(z)| dz$

$\leq d^{(n)}E_np^{(n)}=\frac{d^{(0)}p^{(0)}}{4^n}E_n$

$|\int_{T^{(0)}}^{} f(z) dz|\leq 4^n|\int_{T^{(n)}}^{} f(z) dz|\leq d^{(0)}p^{(0)}E_n$ ($\forall n$)

Take limit on both sides as $n\rightarrow \infty\Rightarrow |\int_{T^{(0)}}^{} f(z) dz|=0$ 

\end{flushleft}
\end{document}