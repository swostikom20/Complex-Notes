\documentclass{article}
\usepackage[utf8]{inputenc}
\usepackage{a4wide}
\usepackage{dsfont}
\usepackage{amsmath}
\usepackage{upgreek}
\usepackage{setspace} \doublespacing
\title{Complex notes}
\author{Om Swostik Mishra}
\date{}
\begin{document}
\maketitle

\begin{flushleft}
\textbf{Defn}: An open, connected subset of $\mathds{C}$ is called domain or region.

$D(a,r)=\{z\in \mathds{C}|\: |z-a|<r\}$

$\{z\in \mathds{C}|\:r<|z-a|<R\}$$\rightarrow$annulus

\textbf{Exponential function}: $exp(z):\mathds{C}\rightarrow\mathds{C}$

$e^z=e^x(cos(y)+isin(y))$

$e^{z_1+z_2}=e^{z_1}.e^{z_2}$, $0\notin Rng(exp(z))$

$|e^{ix}|=1$ ($\forall x\in \mathds{R}$)

$e^{z}=1$ iff $z=2n\pi i$,$n\in \mathds{Z}$

$e^{z_1}=e^{z_2}$ iff $z_1=z_2+2n\pi i$

\textbf{Argument}: 
$\theta=arg(z)$ (Angle made wrt positive real axis)

$arg(z)$ for $z\in \mathds{C}\setminus \{0\}$ is multivalued function.

$arg(z): \mathds{C}\setminus \{0\}\rightarrow (\alpha,\alpha+2\pi]$ or $[\alpha,\alpha+2\pi)$ is well-defined ($\forall \alpha \in \mathds{R}$)

\textbf{Principal value}:$Arg(z)=\theta$  ($-\pi<\theta\leq \pi$)

$arg(z)=\{Arg(z)+2n\pi:n\in \mathds{Z}\}$

\textbf{Complex Log}:For $z\in \mathds{C}\setminus \{0\}$, define $log(z)=log(|z|)+arg(z)$ (multivalued)

Whenever $arg(z)$ is well-defined (i.e $R(arg(z))=[\alpha,\alpha+2\pi)$ or $(\alpha,\alpha+2\pi]$)

$\Rightarrow log(z)$ is well defined

$Log(z)=log(|z|)+i.Arg(z)$: principal log ($log(z)=Log(z)+i2n\pi$)

\textbf{Lemma}: For $z\in \mathds{C}\setminus \{0\}$, the values of log(z) are the complex numbers $\omega$ such that $e^{\omega}=z$

Proof: Do it on your own!

\textbf{Limit}: $f: E\subset\mathds{C} \rightarrow \mathds{C}$

$\lim_{z \to z_0} f(z)=l$ 

$\forall \epsilon>0,\exists \delta>0$ such that $0<|z-z_0|<\delta \Rightarrow |f(z)-l|<\epsilon$ 

$z \in (B(z,z_0)\setminus\{0\})\cap E \Rightarrow f(z)\in B(l,\epsilon)$ 

\textbf{Proposition}: A function $f:E\subset  \mathds{C}\rightarrow \mathds{C}$ has a limit point $l$ as $z\rightarrow z_0$ iff $f(z_n)\rightarrow l$ for every sequence $\{z_n\}\subset E\setminus \{z_0\}$ with $z_n\rightarrow z_0$ as $n\rightarrow \infty$

\textbf{Theorem}: If $f:E\subset \mathds{C} \rightarrow \mathds{C}$ has a limit at $z_0$, then $f$ is bounded near $z_0$

\textbf{Continuity}: $f:E\subset \mathds{C}\rightarrow \mathds{C}$

$f$ is continuous at $z_0\in E$ if for every basic neighbourhood $V$ of $f(z_0)$, there is a basic neighbourhood $U$ of $z_0$ such that $f(U)\subset V$

Let $V=B(f(z_0),\epsilon)$, $U=B(z_0,\delta)$

$z\in B(z_0,\delta)\Rightarrow f(z)\in B(f(z_0),\epsilon)$

$|z-z_0|<\delta \Rightarrow |f(z)-f(z_0)|<\epsilon$

Then, $z\in B(z_0,\delta)\setminus \{z_0\} \Rightarrow f(z)\in B(f(z_0),\epsilon)$

$\lim_{z \to z_0} f(z)=f(z_0)$

$Arg: \mathds{C}\setminus \{0\} \rightarrow (-\pi,\pi]$

$Log(z)=log(|z|)+i.Arg(z)$ (defined on $\mathds{C}\setminus \{0\}$)

$Log(z)$ is continuous on $C\setminus (-\infty,0]$ as $Arg(z)$ is continuous on $\mathds{C}\setminus (-\infty,0]$

\textbf{Defn}: Let $G\subset \mathds{C}$ be a region. A continuous function $f:G\rightarrow \mathds{C}$ is called a branch of logarithm in $G$ if:

$e^{f(z)}=z$ ($\forall z\in G$) ($\Rightarrow 0\notin G$)

$e^{Log(z)}=z$ ($\forall z\in \mathds{C}\setminus \{0\}$)

But $Log(z)$ isn't continuous on $\mathds{C}\setminus\{0\}$

Therefore, $Log(z)$ is a branch of log in $\mathds{C}\setminus (-\infty,0]$ 

$f_k=Log(z)+2k\pi i$, $k\in \mathds{Z}$

$e^{f_k(z)}=z$ (Each $f_k$ is a branch of log in G)

\textbf{Theorem}: Let $f:G\rightarrow \mathds{C}$ be a branch of log. Then $g:G\rightarrow \mathds{C}$ is a branch of log iff $g(z)=f(z)+2k\pi i$ (for some $k\in \mathds{Z}$)

Proof: ($\Rightarrow$) if $f$ is a branch of log, then so is $g$

($\Leftarrow$) Let $g:G\rightarrow \mathds{C}$ be a branch of log. 

Then, (i) $g$ is continuous

(ii) $e^{g(z)}=z$, $\forall z\in G$

Since $f:G\rightarrow \mathds{C}$ is a branch of log, we have, 

(i) $f$ is continuous 

(ii)$e^{f(z)}=z$, $\forall z\in G$

$e^{g(z)}=e^z=e^{f(z)}$ ($\forall z\in G$)

$\Rightarrow g(z)=f(z)+ 2k(z)\pi i$ ($k$ depends on $z$)

$k: G\subset \mathds{C}\rightarrow \mathds{Z}$

$k(z)=\frac{1}{2\pi i}(g(z)-f(z))$ ($k$ is continuous)

Since $G$ is connected, $Img(k)$ is connected  (subset of $\mathds{Z}$)

$\Rightarrow k(z)$ is constant 

Hence, the claim follows.

\textbf{Unit disk in $\mathds{C}$}: $D=\{z\in \mathds{C}: |z|< 1\}$

$0\in D$, hence $D$ cannot be a branch of log. 

$D\setminus \{0\}$ isn't a branch of log. (Why?)

$Log(z): \mathds{C}\setminus (-\infty,0]\rightarrow \mathds{C}$

The half-line is called a branch-cut for any member of $\{Log(z)+2k\pi i: k\in \mathds{Z}\}$

\textbf{Q}: How do you make a branch-cut to define a branch of $Log(z+i-1)$?

\textbf{Power functions}: Let $\alpha \in \mathds{C}$. We define $z^{\alpha}$ to be the multi-valued function:

$z^{\alpha}=e^{\alpha log(z)}$ , $z\neq 0$

$z^{\alpha}=e^{\alpha log(z)}= e^{\alpha(log(|z|)+ arg(z))}= e^{\alpha(Log(z)+2k\pi i)}$  ($k\in \mathds{Z}$) 

$=e^{\alpha Log(z)}.e^{2\pi ik\alpha}$ 

Let $\alpha=n \in \mathds{N}$. Then $z^{\alpha}=e^{nLog(z)}$ which is single-valued.

\textbf{Defn}: Let $\Omega \subset \mathds{C}$ and $f: \Omega \rightarrow \mathds{C}$. Then $f$ is said to be differentiable at $z_0\in \Omega$ if,

$f^{'}(z_0)= \lim_{z \to z_0} \frac{f(z)-f(z_0)}{z-z_0}$ exists 

\textbf{Entire functions}: If a function $f$ is analytic at every point in $\mathds{C}\Rightarrow f$ is entire.

Example: $f(z)\rightarrow$ polynomial in $\mathds{C}$ 

$f(z)=frac{1}{z}$ (differentiable at every point on $\mathds{C}\setminus \{0\}$)

\textbf{Proposition}: If $f$ and $g$ are differentiable at $z_0\in \Omega \subset \mathds{C}$, then so are:

(i)$f+g$, (ii)$fg$, (iii)$f/g$ ($g(z_0)\neq 0$)

\textbf{Chain Rule}: $f:\Omega \rightarrow U$, $g:U\rightarrow \mathds{C}$ are holomorphic at $z_0$ and $f(z_0)$ respectively.
Then $g\circ f:\Omega\rightarrow \mathds{C}$ is holomorphic at $z_0$ and $(g\circ f)^{'}(z_0)=g^{'}(f(z_0)).f^{'}(z_0)$

Example: $f(z)=\bar{z}$ isn't differentiable at any point in $\mathds{C}$.

\textbf{Cauchy-Riemann equations}: To do 

\textbf{Defn}: A function is said to be holomorphic at $z_0$ if it is differentiable in a neighbourhood of $z_0$

A path or a curve is a continuous function, $\gamma:[a,b]\rightarrow \mathds{C}$ 
($Rng(\gamma)\subset \mathds{C}$) 

$\gamma(a)$: initial point of path; $\gamma(b)$: endpoint of path

$[a,b]$: parameter interval 

$\gamma$ is said to be:
\begin{enumerate}
    \item closed if $\gamma(a)=\gamma(b)$
    \item smooth or $C^1$ if $\gamma$ is differentiable and $\gamma^{'}$ is continuous
    \item simple if $\gamma$ is one-one
    \item simple closed if $\gamma(a)=\gamma(b)$ and $\gamma$ is one-one on $(a,b)$
    \item piecewise smooth if there are finitely many points $s_0,s_1 \dots s_n\in [a.b]$ with $a=s_0<s_1<s_2 \dots <s_n=b$ such that the restriction of $\gamma$ to each $(s_i,s_{i+1})$ is smooth.

\end{enumerate}

$-\gamma$ or $\gamma^{-1}$ is defined by $\gamma^{-1}(t)=\gamma(a+b-t)$

$\upphi:[0,1]\rightarrow [a,b]$ defined as: $\upphi(t)=a+(b-a)t$ (one-one and differentiable)

\textbf{Line integral}: $f:[a,b]\rightarrow \mathds{C}$ : continuous

$f=u+iv$, where $u,v:[a,b]\rightarrow \mathds{R}$

Define $\int_{a}^{b}f(t) dt= \int_{a}^{b}u(t) dt+ i \int_{a}^{b} v(t) dt$

\textbf{Properties}:

\begin{enumerate}
    \item $\int_{a}^{b} c.f(t) dt= c.\int_{a}^{b}f(t) dt$
    \item $| \int_{a}^{b}f(t) dt |\leq \int_{a}^{b}|f(t)| dt$
\end{enumerate}
\textbf{Length of a smooth curve}: Let $\gamma:[a,b]\rightarrow \mathds{C}$ be a smooth curve. 

$L(\gamma)= \int_{a}^{b} |\gamma^{'}(t)| dt$
$=\int_{a}^{b} \sqrt{\gamma_1^{'}(t)^2+\gamma_2^{'}(t)^2}  dt$  ($\gamma(t)=\gamma_1(t)+i.\gamma_2(t)$)

If $\gamma:[a,b]\rightarrow C$ is piecewise smooth then $L(\gamma)$ is the sum of the length of its smooth parts.

\textbf{Defn(orientation)}:A curve $\gamma$ is positively oriented if traversed in anti-clockwise direction else is negatively oriented.

\textbf{Examples}:
\begin{enumerate}
    \item $\gamma(t)=r.e^{it}$, ($t\in [0,2\pi]$) ($r>0$: simple, smooth curve); 
    
    $L(\gamma)=\int_{0}^{2\pi} |i.r.e^{it}| dt= r.(2\pi)$
    
    \item $\gamma(t)=e^{it}$, ($t \in [0,4\pi]$): closed, smooth, traverses the unit circle twice in the positive direction
\end{enumerate}
\textbf{Integration over paths}:
$\gamma[a,b]\rightarrow \mathds{C}$ is a smooth curve and $f:\gamma \rightarrow \mathds{C}$: continuous 

\textbf{Defn}: $\int_{\gamma}^{} f(z) dz= \int_{a}^{b} f(\gamma(t)) \gamma^{'}(t) dt=\int_{a}^{b}g(t) dt$ 

($g(t)=f(\gamma(t)) \gamma^{'}(t)$ where $g:[a,b]\rightarrow \mathds{C}$)

Let $[a_1,b_1]$ be any closed interval. Then $\exists \upphi:[a_1,b_1]\rightarrow [a,b]$ (one-one,differentiable and $\upphi(a_1)=a;\upphi(a_2)=b$)

$\upphi[a_1,b_1] \rightarrow \mathds{C}$: smooth 

$\int_{a_1}^{b_1}f(\gamma_1(t)).\gamma_1^{'}(t)dt$ ($=\int_{\gamma_1}^{}f(z) dz$)

$=\int_{a_1}^{b_1}f(\gamma(\upphi(t))).\upphi^{'}(t) dt$
$=\int_{\gamma}^{} f(\gamma(s)) \gamma^{'}(s) ds=\int_{\gamma}^{}f(z) dz$ ($\upphi(t)=s$)

If $\gamma$ is piecewise smooth, the integral can be split into the sum of its smooth components:

if $\gamma=\gamma_1+\gamma_2 \dots +\gamma_n$, then $\int_{\gamma}^{} f = \int_{\gamma_1}^{} f + \dots + \int_{\gamma_n}^{} f$.

Note that $\gamma_i's$ are smooth.

\textbf{Proposition}: If $f$ and $g$ are continuous on a smooth curve $\gamma$, then 
\begin{enumerate}
    \item $\int_{\gamma}^{} \alpha f+\beta g = \alpha \int_{\gamma}^{} f + \beta\int_{\gamma}^{} g$
    \item $\int_{\gamma^{-}}^{} f = - \int_{\gamma}^{} f$
    \item $|\int_{\gamma}^{} f(z) dz| \leq \|f\|_{\infty,\gamma} L(\gamma)$ ($\|f\|_{\infty,\gamma}=sup_{z\in \{\gamma\}}|f(z)|$)
\end{enumerate}

$|\int_{\gamma}^{} f|= |\int_{a}^{b} f(\gamma(t)).\gamma^{'}(t) dt| \leq \int_{\gamma}^{} |f(\gamma(t)).\gamma^{'}(t)| dt
\leq \|f\|_{\infty,\gamma} \int_{a}^{b} |\gamma^{'}(t)| dt$   ($L(\gamma)=\int_{a}^{b} |\gamma^{'}(t)| dt$)
\textbf{Examples}: 

(i)Let $\gamma$ be the arc of a circle of radius 3 ($|z|=3$) from $3$ to $3i$.

Show that:

$$|\int_{\gamma}^{} \frac{z+4}{z^3-1} dz| \leq \frac{21\pi}{52}$$

(ii) $\gamma: |z|=2$  (traverse curve in positive direction)

Prove: $$|\int_{\gamma}^{} \frac{e^z dz}{z^2+1} | \leq \frac{4\pi e^2}{3}$$

\textbf{Fundamental theorem of calculus}: 

If $f:[a,b]\rightarrow \mathds{R}$ has a primitive F, then $\int_{a}^{b} f(x) dx =F(b)-F(a)$ ($F^{'}(x)=f(x), \forall x\in [a,b]$)

\textbf{Definition}: Suppose $G\in \mathds{C}$ be a domain. If a continuous function $f:G\rightarrow \mathds{C}$ has a primitive $F$ on $G$ and if $\gamma$ is a smooth curve in G with initial and terminal points $\omega_1$ and $\omega_2$ respectively, then:

$\int_{\gamma}^{} f = F(\omega_1) - F(\omega_2)$

Proof: Let $[a,b]\in \mathds{R}$ be a parameter interval for $\gamma$ and $\gamma(a)=\omega_1$; $\gamma(b)=\omega_2$

Given $F^{'}(z)=f(z)$   ($\forall z \in G$)

$$\int_{\gamma}^{} f = \int_{a}^{b} f(\gamma(t)).\gamma^{'}(t) dt = \int_{a}^{b} F^{'}(\gamma(t))\gamma^{'}(t) dt$$

$$= \int_{a}^{b} (F \circ \gamma)^{'}(t) dt = F \circ \gamma(b)-F \circ \gamma(a)= F(\omega_2)-F(\omega_1)$$

\textbf{Corollary-1}: If $\gamma$ is a closed curve (smooth), then

$\int_{\gamma}^{} f =0$ (Proof follows from FTC)

\textbf{Corollary-2}: If $f\in H(\Omega)$ for a region $\Omega\in \mathds{C}$ and if $f^{'}=0$ on $\Omega$, then $f$ is a constant function.

Proof: Fix a point $\omega_0\in \Omega$. It suffices to show that $f(\omega)=f(\omega_0), \forall \omega \in \Omega$

\textbf{Simple Closed Curve}: 

Jordan-curve theorem: Every simple closed curve in $\mathds{C}$ divides the complex plane into two regions. One of these regions is bounded and the other is unbounded. The bounded region is called the interior of the curve.

\textbf{Example}: $G=\mathds{C}\setminus \{0\}$

$f(z)=\frac{1}{z}$ on $G$, $\gamma: |z|=1$, $\gamma(t)=e^{it}$, ($t\in [0,2\pi]$)

$$\int_{\gamma}^{} f= \int_{0}^{2\pi} f(\gamma(t)).\gamma^{'}(t) dt= \int_{0}^{2\pi} \frac{i.e^{it}}{e^{it}} dt =2\pi i\neq 0$$

\textbf{Winding number or index of a closed curve}:
Let$\gamma$ be a closed curve on $\mathds{C}$ and let $\alpha \in \mathds{C}\setminus \{\gamma\}$. The winding number of $\gamma$ about $\alpha$ or the index of $\gamma$ with respect to $\alpha$ is denoted by:
$\eta(\gamma;\alpha)/Ind_{\gamma}(\alpha)$ defined by:

$$\eta(\gamma;\alpha)=\frac{1}{2\pi i}\int_{\gamma}^{}\frac{dz}{z-\alpha}$$

\textbf{Example}: $\gamma: [0,6\pi]\rightarrow \mathds{C}$ 

$\gamma(t)=a+re^{it}$ 

$$\eta(\gamma;\alpha)=\frac{1}{2\pi i}\int_{\gamma}^{} \frac{\gamma^{'}(t)}{\gamma(t)-a}=\frac{1}{2\pi i}\int_{\gamma}^{} \frac{1}{a+re^{it}-a}.ire^{it} dt=1$$

\textbf{Theorem}: Let $\gamma$ be a smooth, closed curve in $\mathds{C}$. Let $\alpha\in \mathds{C}\setminus \{\gamma\}$. Then $\eta(\gamma;\alpha)\in \mathds{Z}$.

Proof: To be done


\end{flushleft}
\end{document}
